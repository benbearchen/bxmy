\documentclass{article}

\usepackage{zhfontcfg}
\usepackage[colorlinks,linkcolor=black]{hyperref}
%opening
\title{XeTeX 从 w32tex.org 安装记录}
\author{BenBear}
\begin{document}
\maketitle

\section{下载准备}

XeTeX 可能有很多安装方法,这里使用从 http://w32tex.org/ 下载然后安装的方法。\par

\section{安装过程}
\subsection{创建安装目录}
安装目录可以到某个目录。当然路径名字中没有空格和中文等是安全的。假设要安装到 \mono{C:{\textbackslash}Tools{\textbackslash}XeTex} 。\par

\subsection{解压 XeTeX}
在 XeTeX 目录下创建 \mono{texinst2013} 目录,再把 texinst2013.zip 的文件解压到此目录下。得到可执行文件:\\
\mono{XeTeX{\textbackslash}texinst2013{\textbackslash}texinst2013.exe}\par
接着把需要安装的包(\mono{*.tar.tx})拷到比如 \mono{XeTeX{\textbackslash}pkg} 目录下。然后可以在 cmd 下执行如此命令:\\
\mono{{\textbackslash}> set PATH=\%PATH\%;C:{\textbackslash}Tools{\textbackslash}XeTeX{\textbackslash}texinst2013\\
{\textbackslash}> texinst2013 pkg}\par
然后就可以等待解压完成。解压完成后,会有添加 PATH 路径的提示,一般是要指 \mono{C:{\textbackslash}Tools{\textbackslash}XeTeX{\textbackslash}bin} 目录。添加 PATH 后重新进入 cmd,就可以开始配置字体了。\par

\subsection{更新字体}
以上解压、配置完成以后,就可以扫描字体了。当然首先是安装字体。\par

\subsubsection{安装字体}
包 zhfontcfg.sty 需要几种字体,在资源里已经包含了。把几种字体安装或复制到 Windows 的系统字体目录 \mono{\%Windows\%{\textbackslash}Fonts} 就可以了。\\
如果你有自己钟爱的字体,当然也是一样装起来就好了。\par

\subsubsection{扫描字体}
字体安装好后,在 cmd 执行以下命令:\\
\mono{{\textbackslash}> fc-cache -v}\\
完成后会提示 \mono{success},表示系统字体已经缓存好了。\par

\subsection{更新缺失包}
如果在使用 XeTeX 的过程中发现需要安装当前没有的包,那么把包下载到比如 pkg2 的临时目录,重新执行 \mono{texinst2013 pkg2} 即可(当然要临时修改 PATH)。\par

\section{TeX 测试}
当然我可能喜欢 vim 而没有装专用的 GUI 编辑器,下面说的都是在命令行下编译 PDF。\par

\section{附录}
\begin{enumerate}
\item \mono{http://blog.csdn.net/yming0221/article/details/7410027} \\
LaTeX使用--使用XeLaTeX支持中文(可以放弃Office了) by yming0221 \\
\item \mono{http://electronic-blue.wikidot.com/doc:xetex} \\
XeTeX 快速上手 by electronic\_blue \\
\item \mono{http://w32tex.org/index-zh.html} \\
w32tex 官方中文说明 \\
\end{enumerate}
\end{document}

