\documentclass{article}

\usepackage{zhfontcfg}
\usepackage[colorlinks,linkcolor=black]{hyperref}
\usepackage{color,xcolor}            % 支持彩色文本、底色、文本框等
\usepackage{listings}
\lstset{tabsize=4, keepspaces=true,
    xleftmargin=2em,xrightmargin=0em, aboveskip=1em,
    backgroundcolor=\color{gray!20},  % 定义背景颜色
    frame=none,                       % 表示不要边框
    extendedchars=false,              % 解决代码跨页时,章节标题,页眉等汉字不显示的问题
    basicstyle=\small,
    keywordstyle=\color{black}\bfseries,
    breakindent=10pt,
    identifierstyle=,                 % nothing happens
    commentstyle=\color{blue}\small,  % 注释的设置
    morecomment=[l][\color{blue}]{\#},
    stepnumber=1,numberstyle=\scriptsize,
    showstringspaces=false,
    showspaces=false,
    flexiblecolumns=true,
    breaklines=true, breakautoindent=true,breakindent=4em,
    escapeinside={/*@}{@*/},
}

%\newcommand{\shell}[1]{\colorbox{cyan!15}{\mono{#1}}}
\usepackage{framed}
\newcommand{\shell}[1]{\definecolor{shadecolor}{rgb}{0.5,0.95,0.95} \begin{shaded}\mono{#1}\end{shaded}}

%\parindent=1pt
\setlength{\parskip}{2pt}

\tolerance=1
\emergencystretch=\maxdimen
\hyphenpenalty=10000
\hbadness=10000

\title{\XeTeX 从 w32tex.org 安装记录}
\author{BenBear}
\begin{document}
\maketitle

\section{XeTeX 与 w32tex}

{\XeTeX} 对 Unicode 的支持,让我再次心动,回归 {\TeX} 的世界。{\XeTeX} 可能有很多安装方法,有 TeXLive 这种巨无霸,但也有小巧的 w32tex。这里使用从 http://w32tex.org/ 下载然后安装的方法。目前 w32tex 正好是 2013 版(安装程序的名字看来)。

\section{安装过程}
w32tex 有简单的安装程序,然后再下载若干宏包,就可以安装了。下面逐步说明。

\subsection{创建安装目录}
安装目录可以到某个目录。当然路径名字中没有空格和中文是个安全的选择。假设要安装到 \mono{C:{\textbackslash}Tools{\textbackslash}XeTeX}。

\subsection{下载安装包}
第一个要下载的是简易安装程序,目前叫 \mono{texinst2013.zip}。在 \mono{XeTeX} 目录下创建一个 \mono{pkg} 的目录备用。然后在 w32tex 中文首页(\mono{http://w32tex.org/index-zh.html})挑选挑选一个镜像。找到 \mono{texinst2013.zip} 并下载到 \mono{XeTeX{\textbackslash}pkg} 目录。

除了安装程序,还要下载 {\TeX} 所需要的宏包,依然放到 \mono{pkg} 目录。w32tex 首页有说明从最小安装、标准安装到完全安装所需要的数量不等的宏包。建议是下载最小安装的全部和标准安装中的大部分包(我没有装不认识的 xindy 和 xypic 两个),然后加上完全安装中可能需要包。不“完全”安装是因为安装后的体积问题。选择的 23 个宏包本身占用 383M 空间,解压安装后占用 845M,两者放一起就上 G 了。

w32tex 有对 64 位的支持,也有提供 64 位的宏包下载。不过装了 64 位宏包后,觉得有问题。后来重新安装的时候就不考虑 64 位的事情了。毕竟 32 位也能用得好好的。

以下就是我使用的包列表,编译这篇 {\TeX} 没有问题,虽然不一定是最简的:\\context.tar.xz dvipdfm-w32.tar.xz dvipsk-w32.tar.xz jtex-w32.tar.xz latex.tar.xz ltxpkgs.tar.xz luatex-w32.tar.xz luatexja.tar.xz makeindex-w32.tar.xz manual.tar.xz mftools.tar.xz pdftex-w32.tar.xz platex.tar.xz ptex-w32.tar.xz t1fonts.tar.xz timesnew.tar.xz txpx-pazofonts.tar.xz txtutil.tar.xz utf.tar.xz vf-a2bk.tar.xz web2c-lib.tar.xz web2c-w32.tar.xz xetex-w32.tar.xz

\subsection{解压 XeTeX}
在 \mono{XeTeX} 目录下创建 \mono{texinst2013} 目录,再把 \mono{XeTeX{\textbackslash}pkg{\textbackslash}texinst2013.zip} 包含的文件解压到此目录下。得到可执行文件:\mono{XeTeX{\textbackslash}texinst2013{\textbackslash}texinst2013.exe}。有了这个就可以正式解压安装了。在 cmd 执行如下命令:

\shell{\noindent{\textbackslash}> set PATH=\%PATH\%;C:{\textbackslash}Tools{\textbackslash}XeTeX{\textbackslash}texinst2013\\
{\textbackslash}> texinst2013 pkg}

等解压完成后,会有添加 \mono{\emph{PATH}} 路径的提示,这里应该是指是添加 \mono{C:{\textbackslash}Tools{\textbackslash}XeTeX{\textbackslash}bin} 目录。添加 \mono{\emph{PATH}} 后重新进入 cmd,就可以开始分析字体了。

\subsection{更新字体}
XeTeX 可以使用系统的字体。本例子使用的宏包 zhfontcfg.sty 需要几种字体,可从 \mono{https://github.com/benbearchen/benbeares/tree/master/tex/fonts} 下载。把几种字体安装或复制到 Windows 的系统字体目录 \mono{\%\emph{Windows}\%{\textbackslash}Fonts} 就可以了。如果你有自己钟爱的字体,当然也是一样装起来就好了。字体安装好后,在 cmd 执行以下命令:

\shell{\noindent{\textbackslash}> fc-cache -v}

完成后会提示 success,表示系统字体已经缓存好了。执行 cmd 命令 \mono{fc-list} 可以查看已经扫描到的字体:

\shell{\noindent{\textbackslash}> fc-list | grep YaHei}

\subsection{安装新包或缺失包}
如果在使用 {\XeTeX} 的过程中发现需要安装当前没有的包,那么把包下载到比如 \mono{XeTeX{\textbackslash}pkg2} 的临时目录,重新执行 \mono{texinst2013} 即可:
    
\shell{\noindent{\textbackslash}> set PATH=\%PATH\%;C:{\textbackslash}Tools{\textbackslash}XeTeX{\textbackslash}texinst2013\\
{\textbackslash}> texinst2013 pkg2}

\section{TeX 测试}
当然我可能喜欢 vim 而没有装专用的 GUI 编辑器,下面说的都是在命令行下编译 PDF。
首先拷贝以下 {\TeX} 代码\footnote{代码在中文字符前插入“\mono{- }”是因为 lstlisting 可能对中英文混排支持有问题}到任意目录下的一个 \mono{test.tex} 文件:

%\newpage
\begin{lstlisting}[language={[LaTeX]TeX}]
\documentclass{article}

\usepackage{zhfontcfg}
\usepackage[colorlinks,linkcolor=black]{hyperref}
\title{latex 多字体简易示例}
\author{- 颜开}

\begin{document}
\maketitle
\section{- 楷体}
\kai{- 楷体}
\section{- 黑体}
\hei{- 黑体}
\end{document}
\end{lstlisting}

然后再 cmd 进入 \mono{test.tex} 所在目录,执行命令:

\shell{\noindent{\textbackslash}> xelatex test.tex}

注意是命令名字是 \colorbox{cyan!20}{\mono{xelatex}} 而不是 \mono{xetex}。后者虽然可以运行,但是似乎并不支持 \LaTeX。如果没有错误,命令会直接结束,最后会有一个 PDF 的名字提示——这里会是 \mono{test.pdf}。打开 \mono{test.pdf} 就可以查看输出结果了。

\section{vim 定义快捷键}
为了方便 vim 下使用 {\XeTeX} 编译并打开 PDF 查看,这里还定义了一些简单的 vim 命令与快捷键(需要写入配置文件如 \mono{\_vimrc}):
\begin{lstlisting}
com XeTeX silent !start cmd /c "xelatex -output-directory=%:p:h % && explorer %:p:r.pdf"
nmap <A-S-x> :XeTeX<CR>
com PDFTeX silent !start explorer %:p:r.pdf
\end{lstlisting}

其中 :XeTeX 作为外部命令,执行后会在新控制台窗口调用 xelatex 编译当前的 tex 缓冲区。并在编译无误后,直接用 Windows 的默认软件打开生成的 PDF 文件。而 :PDFTeX 命令则仅打开之前生成的 PDF 文件。对 :XeTeX 定义了一个 Alt-Shift-x 的快捷键,使用更方便。

如果做得更好一点,还需要判断当前缓冲区是否已经保存。保存了的 tex 编译才更安全。

\section{关于本文}
本文除了作为一个 w32tex 的安装记录,也作为 {\XeTeX} 在 vim 中的练习。不足之处,敬请见谅。

本文的 {\XeTeX} 源码可以在 \mono{https://github.com/benbearchen/bxmy/tree/master/article} 找到,包括源文件 \mono{xetex\_w32tex.tex} 与中文宏包 \mono{zhfontcfg.sty}。本文使用的字体可以在 \mono{https://github.com/benbearchen/benbeares/tree/master/tex/fonts} 找到。

{\large{谢谢!}}

\section{参考资料}
\begin{enumerate}
\item \mono{http://blog.csdn.net/yming0221/article/details/7410027} \\
《LaTeX 使用--使用 XeLaTeX 支持中文(可以放弃 Office了)》 \\
by yming0221 {\small 注:此文定义了 zhfontcfg 宏包} %\\
\item \mono{http://electronic-blue.wikidot.com/doc:xetex} \\
《XeTeX 快速上手》 by electronic\_blue \\
\item \mono{http://w32tex.org/index-zh.html} \\
《w32tex 官方中文说明》 \\
\item \mono{https://wiki.freebsdchina.org/doc/x/xelatex} \\
《TeX 中文化的捷径:使用 xelatex》 \\
\item \mono{http://vimcdoc.sourceforge.net/doc/cmdline.html} \\
《VIM 参考手册——命令行模式》 \\
by Bram Moolenaar 翻译 Dasn, tocer  http://vimcdoc.sf.net \\
\item \mono{http://www.freezhongzi.info/?p=90} \\
《Windows 下去掉 Vim 执行外部命令的烦人提示》 by 波波 \\
\end{enumerate}
\end{document}

